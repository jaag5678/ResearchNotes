\documentclass[notes, xcolor = dvipsnames]{beamer}

\usetheme{Warsaw}

\usepackage{amsmath}
\usepackage{graphicx}

\title{Memory Barriers: A Hardware view for Software Hackers}
\subtitle{Paul E. McKenney}

\author{Presented by \\ Akshay Gopalakrishnan}

\begin{document}
  
    \begin{frame}
        
        \maketitle
    \end{frame}


    \begin{frame}{Introduction}
        
        %This paper represents the inner working of hardware that results in several non-sequential behaviors of our concurrent programs
        %The paper is rife with examples as well as showcasing the reasons for having such hardware features which in turn help in our programs performing better.
        %Along with the positives the author also carefully cautions why such rampant changes for performance might result in highly non-trivial behaviors being showcased by the hardware running our programs.
        %The paper concludes by discussing the then versions of several concurrent hardware that exhibit different non-sequential behaviors.

    \end{frame}

    \begin{frame}{Cache structures}
        
    \end{frame}

    \begin{frame}{Th usefullness of Caches}
        
    \end{frame}

    \begin{frame}{THe MIME protocol}
        
    \end{frame}

    \begin{frame}{Example of Cache Communication methods}
        
    \end{frame}

    \begin{frame}{Example}
        
    \end{frame}

    \begin{frame}{Need for Write Buffers}
        
    \end{frame}

    \begin{frame}{Added Complications}
        
    \end{frame}

    \begin{frame}{Example}

    \end{frame}

    \begin{frame}{Here comes Write Memory Barriers}
        
    \end{frame}

    \begin{frame}{Need for Invalidate Queues}
        
    \end{frame}

    \begin{frame}{Added Complications}
        
    \end{frame}

    \begin{frame}{Example}
        
    \end{frame}

    \begin{frame}{Here comes Read Memory Barriers}
        
    \end{frame}

    \begin{frame}{Barrier instructions offered by Linux}
        
    \end{frame}

    \begin{frame}{Hardware Example: Alfa}
        
    \end{frame}

    \begin{frame}{Hardware Example: x86}
    
    \end{frame}

    \begin{frame}{Conclusion}
        
    \end{frame}

\end{document}