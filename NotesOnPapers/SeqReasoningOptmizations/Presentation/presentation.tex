\documentclass[xcolor=dvipsnames, notes]{beamer}
%\documentclass[xcolor=dvipsnames]{beamer}

\usetheme{Warsaw}

\usepackage{amsmath}
\usepackage{amsthm}
\usepackage{graphicx}

\title{Sequential Reasoning for Optimizing Compilers Under Weak Memory Concurrency}
\subtitle{Minki Cho, Sung Hwan Lee, Dongjae Lee, Chung Kil Hur, Ori Lahav}

\author{Presented by \\ Akshay Gopalakrishnan}

\begin{document}

    \begin{frame}
        
        \maketitle

    \end{frame}


    \begin{frame}{Introduction}
        
        \begin{itemize}
            \item Performing/designing thread-local optimizations in a concurrent context would require understanding the underlying memory consistency model.
            \item Often, these models are complex, as they are weak consistency based.
            \item Thus, it becomes difficult to design optimizations which are safe.
            \item Most optimizations that are performed on concurrent programs mainly involve reordering or eliminating non-atomics.
            \item Is there perhaps, then a simple sequential semantics that an optimization designer can rely on?
        \end{itemize}

    \end{frame}

    %Should take about 1hr to do and finish off
    \begin{frame}{Main Idea}
        
    \end{frame}

    

\end{document}