\documentclass[notes, xcolor=dvipsnames]{beamer}

\usetheme{Warsaw}

\usepackage{inputenc}
\usepackage{graphicx}
\usepackage{amsmath}
\usepackage{amsthm}

\title{Understanding POWER Multiprocessors}
\subtitle{Sushmit Sarkar, Peter Sewell, Jade Alglave, Luc Maranget, Derek Williams}

\author{Presented by \\ Akshay Gopalakrishnan}

\begin{document}
    
    \begin{frame}

        \maketitle

    \end{frame}

    \begin{frame}{Introduction}

        \begin{itemize}
            \item Much of the performance in hardware comes due to features such as Read/Write buffers, Speculation, Caches, etc.
            \item The behavior of execution of programs utilizing all such features in a hardware can be defined by a relaxed memory consistency model.
            \item ARM, x86, POWER.. all these hardware exhibit relaxed behaviors.
            \item Of these POWER is relatively well understood.
            \item Major reason is due to informal specification and behaviors described only via litmus tests. 
            \item Continual hardware improvements in POWER also have resulted in more relaxed behaviors.
        \end{itemize}

    \end{frame}

    \begin{frame}{Goal}

        \begin{itemize}
            \item This paper analyzes the POWER multiprocessor family for relaxed behaviors explain their discovered behaviors via an Abstract Machine.
            \item This helps in avoiding getting involved in the complexities of hardware itself while all the way understanding the weak beahviors exhibited by them.
            \item The abstract machine has only one storage subsystem while having read/write buffers for each thread.
        \end{itemize}
        
    \end{frame}

    %Start with some preliminary definitions of orders 
    \begin{frame}{Preliminary Definitions}
        
    \end{frame}

    \begin{frame}{Example 1: Message Passing}

    \end{frame}

    \begin{frame}{Example 2: Store Buffering}
        
    \end{frame}

    \begin{frame}{Example 3: IRIW}

    \end{frame}

    \begin{frame}{Use of Synchronization and other Barriers}
        
    \end{frame}

    %From here on the examples involve sync/data/addr/ctrl dependencies.
    %Make sure to understand their differences to observe the examples in detail.
    \begin{frame}{MP + sync}
        
    \end{frame}



\end{document}