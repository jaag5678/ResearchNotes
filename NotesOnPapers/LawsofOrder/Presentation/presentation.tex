\documentclass[notes, xcolor=dvipsnames]{beamer}

\usetheme{Warsaw}

\usepackage{inputenc}
\usepackage{amsmath}
\usepackage{amsthm}
\usepackage{graphicx}

\title{Laws of Order: Expensive Synchronization in Concurrent Algorithms Cannot be Eliminated}
\subtitle{hagit A, Rachid G, Danny H, Petr K, Maged M, Martin V}

\author{Presented by \\ Akshay Gopalakrishnan}

\begin{document}
    
    \begin{frame}

        \maketitle

    \end{frame}


    \begin{frame}{Introduction}
        \begin{itemize}
            \item Designing Concurrent Algorithms with good performance is a non-trivial tasks.
            \item Major part of slowdown is due to synchronization.
            \item Hence efforts are made to remove such costly synchronizations.
            \item This paper shows that it is impossible to remove completely all expensive synchronization primitives for a class of concurrent implementations.
            \item Such a result helps designers to assert when they can stop improving an algorithm this way.
        \end{itemize}
    \end{frame}

    \begin{frame}{Summary of Results}

        The results are based on the usage of atomic Read-after-Write and Write-after-Read operations. 
        The authors show that without using the above primitives, it is: 
        \begin{itemize}
            \item Impossible to build a linearizable implementation that is non-commutative and respects deterministic sequential specification. 
            \item Impossible to build an algorithm that respects mutual exclusion and is deadlock-free. 
        \end{itemize}
        
    \end{frame}

    \begin{frame}{Mutual Exclusion}
        
        Given $N$ processes, each of which accesses a critical section by acquiring a lock, mutual exclusion requires 
        \center{No more than one process can be in the critical section at the same time}

    \end{frame}

    \begin{frame}{Proof intuition}

        Proof by contradiction.
        Part 1:
        \begin{itemize}
            \item Assume that we do not use a shared write updating the lock. 
            \item Then this would imply more than one thread can be in the critical section. 
        \end{itemize}

        Part 2:
        \begin{itemize}
            \item Assume that we do not use ARAW or AWAR as part of the lock implementation. 
            \item Then this would imply more than one thread can read concurrently the same lock and be ready to acquire the lock (update the lock variable).
            \item Then one thread can acquire the lock (by updating the lock variable). 
            \item After which another thread can also acquire the lock (by updating lock variable too!). 
            \item Thu, violating mutual exclusion.
        \end{itemize}

    \end{frame}

    \begin{frame}{Linearizability}

        An algorithm is linearizable w.r.t a sequential specification if
        \center{Each execution of an algorithm is equivalent to some sequential execution of that specification, where the order between non-overlapping methods is preserved.}
        
        Note: Not all linearizable algorithms need to use RAW/WAR. Only some, which have two properties.

    \end{frame}

    \begin{frame}{Two Properties of Linearizable Algorithms that must use RAW/WAR}
        
        \begin{itemize}
            \item Deterministic Sequential Specification.
            \item Strongly non-commutative methods.
        \end{itemize}

    \end{frame}

    \begin{frame}{Proof Intuition}
        
    \end{frame}

    \begin{frame}{Examples of strongly non-commutative methods}
        %Take two to three key examples
    \end{frame}

    \begin{frame}{Formal proof elements}
        
        \begin{itemize}
            \item Formal language with transition semantics.
            \item Formal definition for executions.
            \item Notion of histories as sub-executions (traces).
            \item Proving both above claims in the same flow using these formal elements. 
        \end{itemize}

    \end{frame}
    
    \begin{frame}{Conclusion}
        
    \end{frame}

    \begin{frame}{Thank you}
        Questions? 
    \end{frame}

\end{document}