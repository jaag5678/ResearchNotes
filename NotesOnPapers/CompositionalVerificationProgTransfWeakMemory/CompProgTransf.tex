\documentclass[xcolor=dvipsnames, notes]{beamer}

\usetheme{Warsaw}

\usepackage{inputenc}
\usepackage{amsmath}
\usepackage{amsthm}
\usepackage{graphicx}

\newcommand{\po}{\textcolor{BlueViolet}{po}}
\newcommand{\rf}{\textcolor{Green}{rf}}
\newcommand{\co}{\textcolor{BurntOrange}{co}}
\newcommand{\mo}{\textcolor{Red}{mo}}
\newcommand{\hb}{\textcolor{NavyBlue}{hb}}
\newcommand{\fr}{\textcolor{RubineRed}{fr}}
\newcommand{\xhb}{\textcolor{NavyBlue}{xhb}}
\newcommand{\rfe}{\textcolor{Green}{rfe}}
\newcommand{\rfi}{\textcolor{Green}{rfi}}
\newcommand{\sw}{\textcolor{BurntOrange}{sw}}
\newcommand{\jhb}{\textcolor{NavyBlue}{jhb}}
\newcommand{\jmo}{\textcolor{Red}{jmo}}
\newcommand{\eco}{\textcolor{WildStrawberry}{eco}}


\title{Compositional Verification of Compiler Optimizations on Relaxed Memory}
\subtitle{Mike Dodds, Mark Batty and Alexey Gotsman}

\author{Presented by \\ Akshay Gopalakrishnan}

\begin{document}
    
    \begin{frame}

        \maketitle
    \end{frame}

    \begin{frame}{Introduction}
        
        %What this paper is about 
        %What it tries to show
        %How in brief
        %Any tools that we can use
        %Proofs ? 

    \end{frame}

    \begin{frame}{Program Transformations as Observational Refinements}
    
        %One line statement of valid transformation
        %One line statement on what is an observation 
        %Observational refinement symbol and statement. 

    \end{frame}

    \begin{frame}{Compositional Reasoning of Tranasformations}

        %Code block and context.
        %Modified definition of valid transformation.

    \end{frame}

    \begin{frame}{Compositional Verification: Idea}

        %Difficult to determine all possible contexts.
        %Create an abstraction to consider only relevant contexts (denotation)
        %Reason about transformations using this abstract context.
        %Prove Soundness and Completeness.

    \end{frame}

%--------------------------------------------------------------

    \begin{frame}{Denotation of Code Blocks: Block Executions} 

        %Denotation is mainly pertaining to the context the block can have

        %Relevant details for each block execution
            %Local variables
            %Global variables -- only concerning ourselves with those global variables that the code block accesses.
            %Context happens-before --- Any happens-before relations that the context has between its events. 
            %Context-atomicity ---> SB edges.

    \end{frame}

    \note{
        We have a code block within which we do some transformation.
        Now, what external stuff can affect its validity? 
        The authors defined this as a context, and we record every relevant infromation (which we call denotation) of a context that can affect the transformation we do on the code block. 
        Note that, the relations we count wrt context does not involve those between context and code block events (like \rf, \hb).
        We need to keep a note of this, as we do not yet have an intuition as to why they aren't considered.
    }

    \begin{frame}{Denotation of Code Blocks: Histories}
        

    \end{frame}

    \begin{frame}{Denotation of Code Blocks: Adequacy and Full Abstraction}


    \end{frame}

%--------------------------------------------------------------------------

    \begin{frame}{Road to Finite Contexts: Reason and Plan}

    \end{frame}

    \begin{frame}{Road to Finite Contexts: Cut Predicate for Contexts}

    \end{frame}

    \begin{frame}{Road to Finite Contexts: Deny Edges and Extended Histories}

    \end{frame}

    \begin{frame}{Road to Finite Contexts: Adequate but lack of Full Abstraction}

    \end{frame}

\end{document}